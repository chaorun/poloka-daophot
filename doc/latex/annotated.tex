\subsection{toads Compound List}
Here are the classes, structs, unions and interfaces with brief descriptions:\begin{CompactList}
\item\contentsline{section}{{\bf Base\-Star}  (The base class for handling stars. Used by all matching routines)}{\pageref{class_basestar}}{}
\item\contentsline{section}{{\bf Component}  (Store the necessary components for weighting and computing the stacking)}{\pageref{class_component}}{}
\item\contentsline{section}{{\bf Counted\-Ref}  (An implementation of \char`\"{}smart pointers\char`\"{} that counts references to an object. The obejct it \char`\"{}points\char`\"{} to has to derive from Ref\-Count)}{\pageref{class_countedref}}{}
\item\contentsline{section}{{\bf Daophot}  (A wrapper to most DAOPHOT routines)}{\pageref{class_daophot}}{}
\item\contentsline{section}{{\bf Db\-Image} }{\pageref{class_dbimage}}{}
\item\contentsline{section}{{\bf Db\-Image\-List}  (Db\-Images can be globally located and stored into image lists)}{\pageref{class_dbimagelist}}{}
\item\contentsline{section}{{\bf DImage}  (A double precision image type)}{\pageref{class_dimage}}{}
\item\contentsline{section}{{\bf Fast\-Finder}  (Fast locator in starlists)}{\pageref{class_fastfinder}}{}
\item\contentsline{section}{{\bf Fits\-Header}  (Fits files and header keys)}{\pageref{class_fitsheader}}{}
\item\contentsline{section}{{\bf Fits\-Image}  (This class enables basic manipulation of images stored in fits files)}{\pageref{class_fitsimage}}{}
\item\contentsline{section}{{\bf Fits\-Image\-Array}  (Array of {\bf Image} {\rm (p.\,\pageref{class_image})} in a FITS file)}{\pageref{class_fitsimagearray}}{}
\item\contentsline{section}{{\bf Fits\-Key}  (Auxilary class for accessing fits header keys)}{\pageref{class_fitskey}}{}
\item\contentsline{section}{{\bf Fits\-Parallel\-Slices}  (Several fits images of the same size to be processed in parallel)}{\pageref{class_fitsparallelslices}}{}
\item\contentsline{section}{{\bf Fits\-Set}  (Container for fits files that have same sizes, filter, and come from the same chip within a mosaic)}{\pageref{class_fitsset}}{}
\item\contentsline{section}{{\bf Fits\-Slice}  (Memory saving traversal of a fits image)}{\pageref{class_fitsslice}}{}
\item\contentsline{section}{{\bf Frame}  (Rectangle with sides parallel to axes)}{\pageref{class_frame}}{}
\item\contentsline{section}{{\bf Fringe\-Utils}  (Utility class for fringe analysis)}{\pageref{class_fringeutils}}{}
\item\contentsline{section}{{\bf Gtransfo}  (A virtual (interface) class for geometric transformations)}{\pageref{class_gtransfo}}{}
\item\contentsline{section}{{\bf Gtransfo\-Cub}  (Implements the cubic transformations (20 real coefficients))}{\pageref{class_gtransfocub}}{}
\item\contentsline{section}{{\bf Gtransfo\-Identity}  (A do-nothing transformation. It anyway has dummy routines to mimick a GTransfo)}{\pageref{class_gtransfoidentity}}{}
\item\contentsline{section}{{\bf Gtransfo\-Lin}  (Implements the linear transformations (6 real coefficients))}{\pageref{class_gtransfolin}}{}
\item\contentsline{section}{{\bf Gtransfo\-Lin\-Rot}  (Just here to provide a specialized constructor, and fit)}{\pageref{class_gtransfolinrot}}{}
\item\contentsline{section}{{\bf Gtransfo\-Lin\-Scale}  (Just here to provide specialized constructors. {\bf Gtransfo\-Lin} {\rm (p.\,\pageref{class_gtransfolin})} fit routine)}{\pageref{class_gtransfolinscale}}{}
\item\contentsline{section}{{\bf Gtransfo\-Lin\-Shift}  (Just here to provide a specialized constructor, and fit)}{\pageref{class_gtransfolinshift}}{}
\item\contentsline{section}{{\bf Gtransfo\-Quad}  (Implements the quadratic transformations (12 real coefficients))}{\pageref{class_gtransfoquad}}{}
\item\contentsline{section}{{\bf Image}  (Class for the basic manipulation of images)}{\pageref{class_image}}{}
\item\contentsline{section}{{\bf Image\-Back}  ({\bf Image} {\rm (p.\,\pageref{class_image})} Back class to compute an image of the background)}{\pageref{class_imageback}}{}
\item\contentsline{section}{{\bf Image\-Gtransfo} }{\pageref{class_imagegtransfo}}{}
\item\contentsline{section}{{\bf Image\-Subtraction}  (For subtracting images using the Alard kernel fit technique. A basic assumption: Ref and New are already geometrically aligned)}{\pageref{class_imagesubtraction}}{}
\item\contentsline{section}{{\bf Image\-Sum}  (A class that handles coadding. Shift and coadd is handled by)}{\pageref{class_imagesum}}{}
\item\contentsline{section}{{\bf Jim\-Star}  (Jim Calibration Star)}{\pageref{class_jimstar}}{}
\item\contentsline{section}{{\bf Kernel}  (An odd size {\bf DImage} {\rm (p.\,\pageref{class_dimage})} addressed with (0,0) at center allows quick computation of convolution like operations)}{\pageref{class_kernel}}{}
\item\contentsline{section}{{\bf Kernel\-Fit}  ({\bf Kernel} {\rm (p.\,\pageref{class_kernel})} fitting by least squares)}{\pageref{class_kernelfit}}{}
\item\contentsline{section}{{\bf Light\-Curve}  (The same Fiducial\-Star monitored many nights)}{\pageref{class_lightcurve}}{}
\item\contentsline{section}{{\bf Light\-Curve\-Builder}  (Lightcurve builder for one band. Execute different types of photometry)}{\pageref{class_lightcurvebuilder}}{}
\item\contentsline{section}{{\bf Light\-Curve\-Guru}  (Main lightcurve builder for all bands)}{\pageref{class_lightcurveguru}}{}
\item\contentsline{section}{{\bf Named\-Value}  (Very simple stuff to associate names and values. Used to I/O transfos to fits headers)}{\pageref{struct_namedvalue}}{}
\item\contentsline{section}{{\bf New\-Sub}  (Handling of an actual subtraction (shift-coadd-subtract-detect). See {\bf Syntax of the \char`\"{}subfile\char`\"{}} {\rm (p.\,\pageref{subfile})} for the way to drive it)}{\pageref{struct_newsub}}{}
\item\contentsline{section}{{\bf Night}  (Version of the {\bf Reduced\-Image} {\rm (p.\,\pageref{class_reducedimage})}, quicker, but with more memory)}{\pageref{class_night}}{}
\item\contentsline{section}{{\bf Night\-Element}  (A template to use when an element belongs to a night)}{\pageref{class_nightelement}}{}
\item\contentsline{section}{{\bf Night\-Set}  (A class representing a set of overlapping images of the same night, same instrument and same filter)}{\pageref{class_nightset}}{}
\item\contentsline{section}{{\bf Point}  (A point in a plane)}{\pageref{class_point}}{}
\item\contentsline{section}{{\bf Psf\-Match}  (A class that wraps calls to {\bf Kernel\-Fit} {\rm (p.\,\pageref{class_kernelfit})}. Used both to carry out subtractions ({\bf Image\-Subtraction} {\rm (p.\,\pageref{class_imagesubtraction})}) and just kernel fitting (for the light curve))}{\pageref{class_psfmatch}}{}
\item\contentsline{section}{{\bf Psf\-Stars}  (A class to select PSF stars from different ways)}{\pageref{class_psfstars}}{}
\item\contentsline{section}{{\bf Reduced\-Image}  (A handle to access data associated to an image: the fits file, the catalog, the dead and satur frames, and a set of 'scalars' such as seeing, saturation level \&co)}{\pageref{class_reducedimage}}{}
\item\contentsline{section}{{\bf SEStar}  (SExtractor star)}{\pageref{class_sestar}}{}
\item\contentsline{section}{{\bf Simultaneous\-Fit}  (Simultaneous fitting of Vignets)}{\pageref{class_simultaneousfit}}{}
\item\contentsline{section}{{\bf Stamp}  (An odd size {\bf DImage} {\rm (p.\,\pageref{class_dimage})} extracted from an {\bf Image} {\rm (p.\,\pageref{class_image})}, centered on (xc,yc))}{\pageref{class_stamp}}{}
\item\contentsline{section}{{\bf Stamp\-List}  (Nothing but a list of Stamps)}{\pageref{class_stamplist}}{}
\item\contentsline{section}{{\bf Star\-List}  (Lists of Stars)}{\pageref{class_starlist}}{}
\item\contentsline{section}{{\bf Star\-Match}  (A pair of stars, usually belonging to different images)}{\pageref{class_starmatch}}{}
\item\contentsline{section}{{\bf Sub}  (Handling of an actual subtraction (shift-coadd-subtract-detect). See {\bf Syntax of the \char`\"{}subfile\char`\"{}} {\rm (p.\,\pageref{subfile})} for the way to drive it)}{\pageref{class_sub}}{}
\item\contentsline{section}{{\bf Sub\-Image}  (A class that allows to cut a subimage from a {\bf Reduced\-Image} {\rm (p.\,\pageref{class_reducedimage})})}{\pageref{class_subimage}}{}
\item\contentsline{section}{{\bf Tan\-Pix2Ra\-Dec}  (The transformation that handles pix to sideral transfos (Gnomonic, possibly with polynomial distortions))}{\pageref{class_tanpix2radec}}{}
\item\contentsline{section}{{\bf Tan\-Ra\-Dec2Pix}  (This one is the Tangent Plane (called gnomonic) projection (from celestial sphere to tangent plane))}{\pageref{class_tanradec2pix}}{}
\item\contentsline{section}{{\bf Transformed\-Image}  (Class that operates the transformation of a {\bf Reduced\-Image} {\rm (p.\,\pageref{class_reducedimage})} (image(s) + list))}{\pageref{class_transformedimage}}{}
\item\contentsline{section}{{\bf Vignet}  (Similar to {\bf Kernel} {\rm (p.\,\pageref{class_kernel})}, but resizeable and contains some kind of {\bf Point} {\rm (p.\,\pageref{class_point})})}{\pageref{class_vignet}}{}
\end{CompactList}
