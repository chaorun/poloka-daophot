\subsection{Frame  Class Reference}
\label{class_frame}\index{Frame@{Frame}}
rectangle with sides parallel to axes. 


{\tt \#include $<$frame.h$>$}

\subsubsection*{Public Methods}
\begin{CompactItemize}
\item 
\index{Frame@{Frame}!Frame@{Frame}}\index{Frame@{Frame}!Frame@{Frame}}
{\bf Frame} ()\label{class_frame_a0}

\item 
\index{Frame@{Frame}!Frame@{Frame}}\index{Frame@{Frame}!Frame@{Frame}}
{\bf Frame} (const {\bf Image} \&image)\label{class_frame_a1}

\begin{CompactList}\small\item\em actual image frame.\item\end{CompactList}\item 
\index{Frame@{Frame}!Frame@{Frame}}\index{Frame@{Frame}!Frame@{Frame}}
{\bf Frame} (const double \&{\bf x\-Min}, const double \&{\bf y\-Min}, const double \&{\bf x\-Max}, const double \&{\bf y\-Max})\label{class_frame_a2}

\begin{CompactList}\small\item\em this one is dangerous: you may swap the 2 middle arguments. Prefer next one.\item\end{CompactList}\item 
\index{Frame@{Frame}!Frame@{Frame}}\index{Frame@{Frame}!Frame@{Frame}}
{\bf Frame} (const {\bf Point} \&Lower\-Left, const {\bf Point} \&Upper\-Right)\label{class_frame_a3}

\begin{CompactList}\small\item\em typical use: Frame(Point(xmin,ymin),Point(xmax,ymax)).\item\end{CompactList}\item 
\index{Frame@{Frame}!Frame@{Frame}}\index{Frame@{Frame}!Frame@{Frame}}
{\bf Frame} (const {\bf Fits\-Header} \&header, Which\-Frame which=Clipped\-Size\-Frame)\label{class_frame_a4}

\begin{CompactList}\small\item\em 2 kinds of bounds in headers, the chip size and some that may be added by hand. See {\bf Write\-In\-Header}() {\rm (p.\,\pageref{class_frame_a22})}.\item\end{CompactList}\item 
\index{Nx@{Nx}!Frame@{Frame}}\index{Frame@{Frame}!Nx@{Nx}}
double {\bf Nx} () const\label{class_frame_a5}

\begin{CompactList}\small\item\em number of pixels in x direction.\item\end{CompactList}\item 
\index{Ny@{Ny}!Frame@{Frame}}\index{Frame@{Frame}!Ny@{Ny}}
double {\bf Ny} () const\label{class_frame_a6}

\begin{CompactList}\small\item\em number of pixels in y direction.\item\end{CompactList}\item 
\index{Center@{Center}!Frame@{Frame}}\index{Frame@{Frame}!Center@{Center}}
{\bf Point} {\bf Center} () const\label{class_frame_a7}

\begin{CompactList}\small\item\em middle of the frame.\item\end{CompactList}\item 
\index{ApplyTransfo@{ApplyTransfo}!Frame@{Frame}}\index{Frame@{Frame}!ApplyTransfo@{Apply\-Transfo}}
Frame {\bf Apply\-Transfo} (const {\bf Gtransfo} \&T) const\label{class_frame_a8}

\begin{CompactList}\small\item\em assumes that Transfo is a shift or involves a 'simple rotation'.\item\end{CompactList}\item 
\index{operator *@{operator $\ast$}!Frame@{Frame}}\index{Frame@{Frame}!operator *@{operator $\ast$}}
Frame {\bf operator $\ast$} (const Frame \&Right) const\label{class_frame_a9}

\begin{CompactList}\small\item\em intersection of Frame's.\item\end{CompactList}\item 
\index{operator *=@{operator $\ast$=}!Frame@{Frame}}\index{Frame@{Frame}!operator *=@{operator $\ast$=}}
Frame\& {\bf operator $\ast$=} (const Frame \&Right)\label{class_frame_a10}

\begin{CompactList}\small\item\em intersection of Frame's.\item\end{CompactList}\item 
\index{operator+@{operator+}!Frame@{Frame}}\index{Frame@{Frame}!operator+@{operator+}}
Frame {\bf operator+} (const Frame \&Right) const\label{class_frame_a11}

\begin{CompactList}\small\item\em union of Frames.\item\end{CompactList}\item 
\index{operator+=@{operator+=}!Frame@{Frame}}\index{Frame@{Frame}!operator+=@{operator+=}}
Frame\& {\bf operator+=} (const Frame \&Right)\label{class_frame_a12}

\begin{CompactList}\small\item\em union of Frames.\item\end{CompactList}\item 
\index{CutMargin@{CutMargin}!Frame@{Frame}}\index{Frame@{Frame}!CutMargin@{Cut\-Margin}}
void {\bf Cut\-Margin} (const double Margin\-Size)\label{class_frame_a13}

\begin{CompactList}\small\item\em shrinks the frame (if Margin\-Size$>$0), enlarges it (if Margin\-Size$<$0).\item\end{CompactList}\item 
\index{CutMargin@{CutMargin}!Frame@{Frame}}\index{Frame@{Frame}!CutMargin@{Cut\-Margin}}
void {\bf Cut\-Margin} (const double Margin\-X, const double Margin\-Y)\label{class_frame_a14}

\begin{CompactList}\small\item\em shrinks the frame (if Margin\-Size$>$0), enlarges it (if Margin\-Size$<$0).\item\end{CompactList}\item 
\index{operator==@{operator==}!Frame@{Frame}}\index{Frame@{Frame}!operator==@{operator==}}
bool {\bf operator==} (const Frame \&Right) const\label{class_frame_a15}

\begin{CompactList}\small\item\em necessary for comparisons (!= is defined from this one implicitely).\item\end{CompactList}\item 
\index{operator"!=@{operator"!=}!Frame@{Frame}}\index{Frame@{Frame}!operator!=@{operator!=}}
bool {\bf operator!=} (const Frame \&Right) const\label{class_frame_a16}

\begin{CompactList}\small\item\em comparison.\item\end{CompactList}\item 
\index{Rescale@{Rescale}!Frame@{Frame}}\index{Frame@{Frame}!Rescale@{Rescale}}
Frame {\bf Rescale} (const double Factor) const\label{class_frame_a17}

\begin{CompactList}\small\item\em rescale it. The center does not move.\item\end{CompactList}\item 
\index{Area@{Area}!Frame@{Frame}}\index{Frame@{Frame}!Area@{Area}}
double {\bf Area} () const\label{class_frame_a18}

\item 
\index{InFrame@{InFrame}!Frame@{Frame}}\index{Frame@{Frame}!InFrame@{In\-Frame}}
bool {\bf In\-Frame} (const double \&x, const double \&y) const\label{class_frame_a19}

\begin{CompactList}\small\item\em inside?\item\end{CompactList}\item 
\index{InFrame@{InFrame}!Frame@{Frame}}\index{Frame@{Frame}!InFrame@{In\-Frame}}
bool {\bf In\-Frame} (const {\bf Point} \&pt) const\label{class_frame_a20}

\begin{CompactList}\small\item\em same as above.\item\end{CompactList}\item 
\index{MinDistToEdges@{MinDistToEdges}!Frame@{Frame}}\index{Frame@{Frame}!MinDistToEdges@{Min\-Dist\-To\-Edges}}
double {\bf Min\-Dist\-To\-Edges} (const {\bf Point} \&P) const\label{class_frame_a21}

\begin{CompactList}\small\item\em distance to closest boundary.\item\end{CompactList}\item 
\index{WriteInHeader@{WriteInHeader}!Frame@{Frame}}\index{Frame@{Frame}!WriteInHeader@{Write\-In\-Header}}
void {\bf Write\-In\-Header} ({\bf Fits\-Header} \&Head) const\label{class_frame_a22}

\begin{CompactList}\small\item\em write a frame in the fits header.\item\end{CompactList}\item 
\index{dump@{dump}!Frame@{Frame}}\index{Frame@{Frame}!dump@{dump}}
void {\bf dump} (ostream \&stream=cout) const\label{class_frame_a23}

\end{CompactItemize}
\subsubsection*{Public Attributes}
\begin{CompactItemize}
\item 
\index{xMin@{xMin}!Frame@{Frame}}\index{Frame@{Frame}!xMin@{x\-Min}}
double {\bf x\-Min}\label{class_frame_m0}

\begin{CompactList}\small\item\em coordinate of boundary.\item\end{CompactList}\item 
\index{xMax@{xMax}!Frame@{Frame}}\index{Frame@{Frame}!xMax@{x\-Max}}
double {\bf x\-Max}\label{class_frame_m1}

\begin{CompactList}\small\item\em coordinate of boundary.\item\end{CompactList}\item 
\index{yMin@{yMin}!Frame@{Frame}}\index{Frame@{Frame}!yMin@{y\-Min}}
double {\bf y\-Min}\label{class_frame_m2}

\begin{CompactList}\small\item\em coordinate of boundary.\item\end{CompactList}\item 
\index{yMax@{yMax}!Frame@{Frame}}\index{Frame@{Frame}!yMax@{y\-Max}}
double {\bf y\-Max}\label{class_frame_m3}

\begin{CompactList}\small\item\em coordinate of boundary.\item\end{CompactList}\end{CompactItemize}
\subsubsection*{Friends}
\begin{CompactItemize}
\item 
\index{operator<<@{operator$<$$<$}!Frame@{Frame}}\index{Frame@{Frame}!operator<<@{operator$<$$<$}}
ostream\& {\bf operator$<$$<$} (ostream \&stream, const Frame \&Right)\label{class_frame_l0}

\begin{CompactList}\small\item\em allows \footnotesize\begin{verbatim} cout << frame; \end{verbatim}\normalsize 
.\item\end{CompactList}\end{CompactItemize}


\subsubsection{Detailed Description}
rectangle with sides parallel to axes.

when Frame's are used to define subparts of images, x\-Min and x\-Max refer to the first and last pixels in the subimage 



The documentation for this class was generated from the following file:\begin{CompactItemize}
\item 
{\bf frame.h}\end{CompactItemize}
