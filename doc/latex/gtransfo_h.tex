\subsection{gtransfo.h File Reference}
\label{gtransfo_h}\index{gtransfo.h@{gtransfo.h}}
Geometrical transformations (of 2D points). 


{\tt \#include $<$iostream$>$}\par
{\tt \#include $<$iomanip$>$}\par
{\tt \#include $<$fstream$>$}\par
{\tt \#include $<$string$>$}\par
{\tt \#include \char`\"{}point.h\char`\"{}}\par
{\tt \#include \char`\"{}rootstuff.h\char`\"{}}\par
{\tt \#include $<$vector$>$}\par
\subsubsection*{Compounds}
\begin{CompactItemize}
\item 
struct {\bf Gtransfo\-Cub::Cub\-Assoc}
\item 
class {\bf Gtransfo}
\begin{CompactList}\small\item\em a virtual (interface) class for geometric transformations.\item\end{CompactList}\item 
class {\bf Gtransfo\-Cub}
\begin{CompactList}\small\item\em implements the cubic transformations (20 real coefficients).\item\end{CompactList}\item 
class {\bf Gtransfo\-Identity}
\begin{CompactList}\small\item\em A do-nothing transformation. It anyway has dummy routines to mimick a GTransfo.\item\end{CompactList}\item 
class {\bf Gtransfo\-Lin}
\begin{CompactList}\small\item\em implements the linear transformations (6 real coefficients).\item\end{CompactList}\item 
class {\bf Gtransfo\-Lin\-Rot}
\begin{CompactList}\small\item\em just here to provide a specialized constructor, and fit.\item\end{CompactList}\item 
class {\bf Gtransfo\-Lin\-Scale}
\begin{CompactList}\small\item\em just here to provide specialized constructors. {\bf Gtransfo\-Lin} {\rm (p.\,\pageref{class_gtransfolin})} fit routine.\item\end{CompactList}\item 
class {\bf Gtransfo\-Lin\-Shift}
\begin{CompactList}\small\item\em just here to provide a specialized constructor, and fit.\item\end{CompactList}\item 
class {\bf Gtransfo\-Quad}
\begin{CompactList}\small\item\em implements the quadratic transformations (12 real coefficients).\item\end{CompactList}\item 
struct {\bf Named\-Value}
\begin{CompactList}\small\item\em very simple stuff to associate names and values. Used to I/O transfos to fits headers.\item\end{CompactList}\item 
class {\bf Tan\-Pix2Ra\-Dec}
\begin{CompactList}\small\item\em the transformation that handles pix to sideral transfos (Gnomonic, possibly with polynomial distortions).\item\end{CompactList}\item 
class {\bf Tan\-Ra\-Dec2Pix}
\begin{CompactList}\small\item\em This one is the Tangent Plane (called gnomonic) projection (from celestial sphere to tangent plane).\item\end{CompactList}\end{CompactItemize}
\subsubsection*{Functions}
\begin{CompactItemize}
\item 
\index{operator<<@{operator$<$$<$}!gtransfo.h@{gtransfo.h}}\index{gtransfo.h@{gtransfo.h}!operator<<@{operator$<$$<$}}
ostream\& {\bf operator$<$$<$} (ostream \&stream, const {\bf Gtransfo} \&T)\label{gtransfo_h_a0}

\begin{CompactList}\small\item\em allows 'stream $<$$<$ Transfo;' (by calling T.dump(stream)).\item\end{CompactList}\item 
\index{GtransfoCompose@{GtransfoCompose}!gtransfo.h@{gtransfo.h}}\index{gtransfo.h@{gtransfo.h}!GtransfoCompose@{Gtransfo\-Compose}}
{\bf Gtransfo}$\ast$ {\bf Gtransfo\-Compose} (const {\bf Gtransfo} $\ast$Left, const {\bf Gtransfo} $\ast$Right)\label{gtransfo_h_a1}

\begin{CompactList}\small\item\em Returns a pointer to a composition. if Left-$>$Reduce\-Compo(Right) return NULL, builds a Gtransfo\-Composition and returns it. deletion of returned value to be done by caller.\item\end{CompactList}\item 
\index{IsIdentity@{IsIdentity}!gtransfo.h@{gtransfo.h}}\index{gtransfo.h@{gtransfo.h}!IsIdentity@{Is\-Identity}}
bool {\bf Is\-Identity} (const {\bf Gtransfo} $\ast$a\_\-transfo)\label{gtransfo_h_a2}

\begin{CompactList}\small\item\em Shorthand test to tell if a transfo belongs to the {\bf Gtransfo\-Identity} {\rm (p.\,\pageref{class_gtransfoidentity})} class.\item\end{CompactList}\item 
\index{operator *@{operator $\ast$}!gtransfo.h@{gtransfo.h}}\index{gtransfo.h@{gtransfo.h}!operator *@{operator $\ast$}}
{\bf Gtransfo\-Quad} {\bf operator $\ast$} (const {\bf Gtransfo\-Lin} \&L, const {\bf Gtransfo\-Quad} \&R)\label{gtransfo_h_a3}

\item 
\index{GtransfoToLin@{GtransfoToLin}!gtransfo.h@{gtransfo.h}}\index{gtransfo.h@{gtransfo.h}!GtransfoToLin@{Gtransfo\-To\-Lin}}
{\bf Gtransfo\-Lin}$\ast$ {\bf Gtransfo\-To\-Lin} (const {\bf Gtransfo} $\ast$transfo)\label{gtransfo_h_a4}

\begin{CompactList}\small\item\em probably obsolete. use Linear\-Approximation instead.\item\end{CompactList}\end{CompactItemize}


\subsubsection{Detailed Description}
Geometrical transformations (of 2D points).



